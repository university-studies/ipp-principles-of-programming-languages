%Encoding: utf8
%Author: Pavol Loffay, xloffa00@stud.fit.vutbr.cz
%Date: 24.02.2012
%Project: Dokumentacia k projektu do predmetu IPP,
%Dokumenta k zadaniu JSON - prevod formatu JSON do XML


\documentclass[10pt,a4paper]{article}

%preambule
\usepackage[czech]{babel}
\usepackage[top=2.5cm, left=1.5cm, text={18cm, 25cm}]{geometry}
\usepackage[IL2]{fontenc}
\usepackage[utf8]{inputenc}

\newcommand\czuv[1]{\quotedblbase #1\textquotedblleft}


%textova cast
\begin{document}

\noindent Dokumentace úlohy JSN: JSON2XML v~Perl do IPP 2011/2012 \\
Jméno a~přijmení: Pavol Loffay \\
Login: xloffa00

\setcounter{section}{1}
\subsection{Úvod}
Tento dokument stručně popisuje převod z~formátu JSON\footnote{Popsán
v~RFC 4627} do XML\footnote{Dostupné na http://www.w3.org/TR/REC-xml/}
v~programovacím jazyce \emph{Perl}, s~využitím níže uvedených knihoven. Dále 
popisuje hlavní problémy, se~kterými jsem se při implementaci setkal.

\subsection{Návrh řešení problému} 
Hlavním implementačním problémem bylo správné načítání dat ve formátu
JSON a~identifikování datových struktur. Následná konverze do formátu
XML byla jednoduchá. Po analýze problému jsem se rozhodl použít knihovny pro práci 
s~daty ve formátu JSON a~XML. Protože skript můžeme spouštět s~různými parametry,
použil jsem knihovnu \texttt{Getopt::Long}, která zpracování
parametrů značně ulehčuje.

\subsection{Zpracování vstupního souboru}
Pro korektní načítání vstupního textu jsem použil knihovnu \texttt{JSON::XS}.
Jednou z~její výhod je, že automaticky kontroluje správnost formátu
vstupních dat. Proto je nebylo potřebné explicitně ověřovat. 
Po načtení celého vstupu do proměnné, bylo zapotřebí identifikovat datové struktury
JSON. Pokud byl načten objekt, tak v~proměnné byl uložen jako 
hash. Pokud bylo načtené pole, bylo uloženo opět do pole. Ostatní prvky bylo možno 
rozlišit jednoduchým způsobem pomocí větvení programu.

\subsubsection{Rozlišení řetězce od čísla}
Skript má parametry, které umožňují transformaci hodnot dvojic typu
\texttt{string} nebo \texttt{number} z~atributů na textové elementy. Proto 
bylo zapotřebí tyto dva typy explicitně rozlišovat. K~tomu jsem použil
modul \texttt{Data::Dumper}, který v~případe ak premňenná byla typu \texttt{string}
tak k~ní přidá uvozovky.
%Číslo jsem identifikoval vynásobením předpokladné
%proměnné pro uložení čísla jedničkou a~následným porovnáním operátorem \texttt{eq}
%se stejnou proměnnou před vynásobením.

\subsubsection{Identifikování \texttt{true}, \texttt{false}, \texttt{null}}
Zvolením správnych parametrů skriptu je možné docílit, aby hodnoty literálů
\texttt{true}, \texttt{false} a~\texttt{null} byly transformovány na elementy
\texttt{<true/> <false/> <null/>}. Proto bylo nutné rozlišovat tyto hodnoty, které
\emph{Perl} nerozlišuje. Hodnoty typu \texttt{Bool} bylo možné identifikovat
pomocí funkce \texttt{JSON::XS::is\_bool}, \texttt{null} jsem rozlišil funkcí \texttt{defined}.

\subsection{Vytváření výstupu}
%Pro vytvoření výstupního XML souboru jsem použil knihovnu \texttt{XML::Writer}.
%To mi umožnilo efektivně vypisovat XML značky. Metody z~této knihovny
%automaticky nahrazují problematické znaky jako jsou $\&$, $<$ a~$>$ na entity 
%$\&amp;$,\ \ $\&lt;$ \dots. V případě, že nebyl zadán parametr \emph{-c}, bylo nutné
%vypisovat retěžce jiným způsobem tak, aby nedocházelo k~vícenásobnému nahrazování.
Pro vytvoření výstupního XML souboru jsem implementoval modul \texttt{xml\_writer}.
Obsahuje metody pro vypisování \textbf{start, end} a~\textbf{empty} XML značek.
Tyto metody mi umožnili efektivně vytvářet výsledný soubor.


\subsection{Kontrola správnosti XML značky}
Pro kontrolu XML značky bylo nutné vytvořit netriviální regulární výraz, který
dokázal rozpoznat nevalidní XML značku. Potom se dalším regulárním výrazem
nahradili všechny nepovolené znaky za \texttt{-}. Poté byla opět provedena kontrola
validnosti značky.

\subsection{Převod JSON do XML}
Samotná konverze formátu probíhá ve~funkci \texttt{json2xml}, která se volá
rekurzivně. Vstupu je načten do proměnné a~následně procházen v~této
funkci. Dále následuje rozpoznávání typu proměnné, která může být hash, pole
atd.. Pokud byla typu hash, nebo pole je funkce pro každou položku opět volaná rekurzivně.
Vyžitím rekurze se daly jednoduchým způsobem identifikovat veškeré zanořené datové 
struktury formátu JSON.

\subsection{Rozšírení JPD}
Implementace tohoto rozšíření spočívala v~tom, že bylo zapotřebí zjistit minimální
možný počet nul, které bylo zapotřebí přidat ke každému indexu tak, aby měly všechny indexy
stejný počet cifer. Nejprve bylo nutné zjistit kolik cifer bude mít poslední
index pole, následně se pak vypočítal počet cifer aktuálního indexu a~provedla se
korekce přidáním správného počtu nul.

\end{document}

