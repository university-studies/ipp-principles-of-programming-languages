%Encoding: utf8
%Author: Pavol Loffay, xloffa00@stud.fit.vutbr.cz
%Date: 14.04.2012
%Project: Dokumentacia k projektu do predmetu IPP,
%Dokumentacia k zadaniu SYN - zvyraznenie syntaxe

%preambule
\documentclass[10pt,a4paper]{article}
\usepackage[czech]{babel}
\usepackage[top=2.5cm, left=2cm, text={17cm, 24cm}]{geometry}
\usepackage[IL2]{fontenc}
\usepackage[utf8]{inputenc}

\newcommand\czuv[1]{\quotedblbase #1\textquotedblleft}

%textova cast
\begin{document}

\noindent Dokumentace úlohy SYN: Zvýraznění syntaxe v~Python 3 do IPP 2011/2012 \\
Jméno a~přijmení: Pavol Loffay \\
Login: xloffa00

\setcounter{section}{1}
%\subsection{Úvod}
%Tento dokument popisuje spôsob implementácie skriptu pre automatické zvýraznenie
%rôznych častí textu pomocou regulárnych výrazov.

\subsection{Analýza úlohy}
V~této úloze jsem na začátku identifikoval tyto významné úlohy.
Zpracování parametrů skriptu
a~formátovacího souboru, kontrola regulárních výrazů a~úprava pro jazyk 
\emph{Python 3} a~následná aplikace regulárních výrazů
a~vkládání formátovacích značek do výsledného textu.

\subsection{Návrh řešení problému}
Úlohu jsem rozdělil do dvou častí. První část zahrnuje zpracování 
formátovacího souboru s~kontrolou a~úpravou regulárních výrazů.
Druhá část se stará o~nalezení indexu, na který se mají vložit 
formátovací řetězce a~jejich následné vkládání.

Další kapitoly popisují jednotlivé problémy. Jejich pořadí je pro
korektní běh skriptu nutné dodržet.

\subsubsection{Zpracování parametrů}
Skript je možné ovládat různými parametry, proto jsem se rozhodl
použít modul \texttt{getopt}, který jejich zpracování výrazně ulehčuje.

\subsection{Popis řešení}
\subsubsection{Zpracování formátovacího souboru}
Po otevření souboru se jeho obsah načítá po řádcích do proměnné pomocí metody
\texttt{readlines()}. Tato metoda vrací seznam, kde každou položkou je jeden řádek. Pro odlišení 
regulárního výrazu od seznamu formátovacích parametrů jsem použil metodu
\texttt{split()}, která rozdělí řetězec na podřetězce podle zadaného znaku.
Tato metoda znova vrací seznam, kdy v~tomto seznamu se regulární výraz vždy 
nachází v~prvním prvku. Zpracování formátovacích příkazů sa provádí
analogickým způsobem. 

Formátovací příkazy jsou ihned transformovány na 
ekvivalentní HTML značky, přičemž se vytvářejí začáteční a~koncové,
takže jsou vždy vytvořeny dva řetězce. 
Jeden, který obsahuje počáteční 
značky a~druhý koncové.
Tímto jsem docílil, že ke~každému regulárnímu výrazu jsou připravené řetězce,
které se vloží na počáteční a~koncovou pozici.

\subsubsection{Kontrola regulárního výrazu}
Z~důvodu, že uživatel může zadat nesprávný regulární výraz, je nutné jej
kontrolovat. Samotné ověření výrazu spočívá v~kontrole, zda-li jsou speciální symboly
\texttt{.|!*+()\%},\ zadané správným způsobem. Kontrola se provádí,
vyhledáváním zakázaných kombinací výše uvedených znaků. Některé se také
nesmí nacházet na začátku, nebo na konci výrazu. Dále bylo nutné ověřit,
zda-li jsou všechny závorky ukončeny.

\subsubsection{Úprava regulárního výrazu}
Po kontrole správnosti regulárního výrazu jej bylo nutné převést na 
ekvivalentní výraz pro jazyk \emph{Python 3}. Nejprve byl celý escapován 
metodou \texttt{escape()}. Následně se zpětně od-escapovali znaky 
\texttt{|*+()}, protože mají shodnou sémantiku. Úprava ostatních častí 
regulárního výrazu spočívala v~záměně některých znaků. Například \texttt{!A}, 
za \texttt{[\textasciicircum A]} nebo odstranění všech výskytů
znaku '\texttt{.}'. 

Kontrola a~úprava regulárních výrazů probíhá
v~části, kde se zpracovává formátovací soubor. 

\subsubsection{Aplikace regulárních výrazů}
Upravené regulární výrazy, byly následně použity pro vyhledávání částí 
vstupního textu, které mají být označeny. Pro vyhledávání jsem použil metodu
\texttt{search()}, která vrátí objekt, na který se dále aplikují následující 
metody. Pomocí metod \texttt{start()} a~\texttt{end()} se zjistí pozice,
na které se mají vložit formátovací řetězce. Nalezené pozice se ukladají do tabulky
s~rozptýlenými položkami, kde klíč je index v~textu, který je označován
a~hodnota je řetězec, který se na daný index vloží. Počáteční HTML značka se pak
vloží na počáteční index a~koncová na koncový index.

\subsubsection{Vkládaní formátovacích značek}
Formátovací značky s~indexy, na které mají být vloženy jsou uloženy
v~tabulce s~rozptýlenými položkami. HTML značky je nutné vkládat do textu od 
posledního indexu k~prvnímu. Proto se vytvoří pomocný seznam, který obsahuje
pouze klíče tabulky. Ten se následně seřadí a~může dojít k~vkládaní
formátovacích řetězců do výsledného textu.

\subsubsection{Parameter \texttt{--br}}
Vkládaní značky \texttt{<br />} na konec řádku probíhá jako poslední fáze
úpravy výsledného textu.

\end{document}

